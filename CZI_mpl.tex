\documentclass[11pt]{article}  % min is 11pt

\usepackage[margin=1in]{geometry}  % min is .5 in
\usepackage[shortlabels]{enumitem}
\usepackage{graphicx}
\usepackage{caption}
\usepackage{footmisc}
\usepackage[hidelinks]{hyperref}



\begin{document}
\title{Matplotlib - Foundation of Visualization in Python}
\author{}
\maketitle

\section{Goals}

The purpose of this proposal is to plan Matplotlib's evolution to meet
the visualization challenges of the coming decade and reduce backlog
of currently open Issues and Pull Requests.

By most measures Matplotlib is a wildly successful project; the source
has been actively developed and maintained by a vibrant, primarily
volunteer, community over the last 16 years and, conservatively, has
over a million users across the wide spectrum of fields, including
bio-medical imaging, microscopy, and genomics
\cite{Carpenter2006,Wolf2018,10.7717/peerj.453}
\cite{Segata2011,10.1371/journal.pgen.1000695,HASHIMSHONY2012666,
  10.1093/bioinformatics/bts480,Carlile2014,Laganowsky2014,Jiangaac9462,
  10.3389/fninf.2014.00014}.  We expect our user base to continue to
grow as Python is adopted by more scientists in the life sciences.
However, given the scale, scope, and importance of the project, we are
at the limit of what can be developed and maintained with primarily
volunteer effort.  This proposal identifies three key areas where we
need sustained support to ensure the health of the platform and
provide the leadership to meet the visualization challenges of the
next 16 years:

\begin{enumerate}[label=\alph*)]
  \item Maintenance of the library, including curating new and
    existing Issues and Pull Requests.
  \item Developing a comprehensive plan to evolve the core architecture
    of Matplotlib.
  \item Developing the tools, documentation, and community to foster a
    rich eco-system of domain-specific plotting tools built on
    Matplotlib.
\end{enumerate}


% The goal is the maintenance; curating is the method of achieving it.
\subsection{Maintenance of the library}

New Issues and Pull Requests are being submitted faster than our
volunteers can review them.
Over the past few years we have merged PRs and closed
issues at about about [X per month], but about [Y] new issues and PRs
are opened monthly; currently we have about 1200 open issues and
300 open PRs.
Among the latter are good contributions and bug fixes
that, possibly with additional attention and polish, could improve
Matplotlib for direct users and downstream packages.
The backlog is
discouraging for new or occasional contributors, and distracting for
core developers.

To maintain Matplotlib's health we need to do the following:

\begin{itemize}
\item Sort the current backlog of Issues and Pull Requests
  in terms of urgency and difficultly.
\item Ensure that newly opened Issues are sorted and Pull Requests
  are reviewed in a timely manner.
\item Address critical bugs and regressions in a timely manner.
\item On-board new contributors to the project team.  This is
  critical to diversifying and sustaining our developer community.
\item Maintain backward compatibility as much as practical.  If we do
  break API, ensure it is intentional, justified, and well documented.
\item Manage group decisions about proposed enhancements, features, and
  breaking API changes.
\end{itemize}

None of this is to demote the importance of the volunteer
contributors, but instead to better co-ordinate and nurture their
efforts, with the goal of growing and sustaining a diverse community
of expert contributors.
% Realistically, don't we need *some* funding for paid work to
% continue after this?  I don't think this is a one-shot cure-all.
% Agree, this is at best a jump start / down payment on future funding

\subsection{Road-map and Architecture }

The current
architecture\footnote{https://www.aosabook.org/en/matplotlib.html} of
Matplotlib was developed 15 years ago \cite{Hunter:2007}.  That it is
still in use is a testament to its initial design; but that design
does not reflect recent developments in data structures, software
design, and visualization.  Matplotlib does not natively know how to
exploit structured (e.g. \texttt{pandas} or \texttt{xarray}) or
streaming data.  We will investigate how to update Matplotlib's
internal data model to use the information embedded in structured
data and allow fluid updating.
While there are many downstream domain-specific libraries that are built on
top of Matplotlib, interoperability between them is problematic.
We will develop
plans for refactoring the core library to facilitate its extension and
to smooth interoperability among the core library and the various
domain-specific plotting tools.

\subsubsection{Homogenization of the API}

The library has grown organically over time through the contributions
of many people (approx. 900 individuals) and the code has accumulated
many small inconsistencies in the API.  Similar methods have different
argument order, e.g., \texttt{ax.text(x, y, s)} vs
\texttt{ax.annotation(s, (x, y))}, and some keyword arguments can be
singular or plural, e.g., \texttt{color} vs \texttt{colors}.  These
subtle issues add friction for users, but are hard to fix without
breaking existing code somewhere in our large user base.  Our goal is
to minimize breakage, but still simplify the API.  Taking into account
\textbf{all} of the APIs, we will have to carefully consider which to
leave as they are, which to deprecate, and which to replace.

\subsubsection{API generalization}

Currently Matplotlib has two main user-facing APIs: the
\texttt{pyplot} API and the \texttt{Object Oriented} (\texttt{OO})
API.
The \texttt{pyplot} API closely follows MATLAB and, while
convenient for quick interactive usage, it is built around the concept
of a global ``current Figure'' and ``current Axes''.
This is
problematic if used in down stream libraries by leading to surprising
and frequently undesired coupling between different parts of the code.
On the other hand the \texttt{OO} interface is more explicit and
flexible but marginally more verbose.
  However, the main name space
for plotting methods is on the \texttt{Axes} object which leads to
three issues with the API.
  First because the plotting methods
provided by Matplotlib are methods on one of our classes third party
domain-specific packages can never feel ``First Class'' as they will
not be implemented as \texttt{Axes} methods.
  Second, there are some
plots that should be easy that require putting \texttt{Artist}s on
multiple \texttt{Axes} which is not something that can be naturally
expressed as an \texttt{Axes} method.
Lastly, because all of the
plotting methods (along with some additional \texttt{Axes} specific
methods) are in a single name space, there are over 250 methods on the
\texttt{Axes} class which makes it extremely hard to discover if the
method you need exists.


To address both of these issues we will move the main plotting name
space from the \texttt{Axes} method to top-level functions that take in
data, style, and \texttt{Axes}s.
This will allow ``first-party'' and
``third-party'' plotting tools to be on a level playing field and
makes it possible to cleanly write plotting tools that produce
multiple \texttt{Artist}s across multiple \texttt{Axes}s.
We will ensure that all new functions have consistent naming and call
conventions.
These
functions will return the rich composites discussed in the next
section.
Having all of the plotting functions, both first- and third-party, as
top level functions allows for curated domain-specific name spaces,
which will aid in the discoverabilty plotting functions by users.

\subsubsection{Rich Composite \texttt{Artist}s}

\texttt{Artist}s are the ``middle layer'' of Matplotlib that encode
user-intent, style, and data.  An \texttt{Artist} can ``draw'' its
self when \texttt{Figure} is rendered to screen or disk.
To update the style or data users interact with the  \texttt{Artist}s.
Currently, Matplotlib has a mix of ``primitive'' \texttt{Artist}s (for
things like lines, images, and patches) and ``composite'' artists for
things like the whole \texttt{Figure}.
In some cases the mapping between the user API and the resulting
\texttt{Artist}s is one-to-one, but in other cases one user call may
generate many decoupled \texttt{Artist}s, for example,
\texttt{errorbar} creates up to 5 independent \texttt{Artist}s or
\texttt{hist} which returns many independent rectangles.  In either
case, if the user wants to update the plot they must touch all of the
\texttt{Artists}.
Instead, we should be providing the user with a single object on which
they can interact with to elide the underlying details of how the
\texttt{Artist} is rendered.
This will simplify user code for interactive data exploration,
handling streaming data, and generating animations.




\subsubsection{Data Model}

``Structured data'' combines multiple pieces of, possibly
heterogeneous, along with labels, dimensions, units, and metadata into
single data structure.
In addition to the data itself, these objects carry information about
the relationship, implicitly or explicitly, between the components.
Currently Matplotlib asks the users or downstream libraries to split
this information into its components and pass them individually to
plotting methods, losing the structure and relationships.
Internally, each plotting method (and \texttt{Artist}s, the objects
used to represent the plot elements) each handle sanitizing and
storing user input independently.
This means that some common functionality, such as handling data with
attached units (e.g., degrees Celsius), is scattered throughout the
code base.
This leads to inconsistencies in behavior across the library and makes
it difficult to write code that updates the user data, as is required
for interactive exploration, streaming, and animation use cases.


We will re-organize the internal data representation in Matplotlib.  The
goal is to have a simple model appropriate for the base Matplotlib
library and, more importantly, to have the technical underpinning to
allow domain-specific downstream libraries to handle and update
structured data in a coherent fashion.
By removing the direct data storage from the \texttt{Artist}s and
defining an API for data objects we will make it easy to:
\begin{itemize}
  \item natively consume structured data;
  \item implement smart down sampling of plotted data based on view
    limits;
  \item seamlessly update the underlying data, either
    streaming or interactively;
  %% this one maybe too much, cool idea, but not really mentioned anywhere else
  \item and back the plot with non-numpy arrays, including queries to
    a database.
\end{itemize}
By defining a clear API to access data we will be able to decouple
the development of the data storage from the \texttt{Artist}s.
This will enable implementations of the data layer that have exotic
dependencies to be used with \texttt{Artist}s from the core library,
and third-party \texttt{Artist}s to rely on the data layer from the
core library.



\subsubsection{Additional Export Methods}

The primary way to export a Matplotlib figure is to render it to
either a raster or vector file format.  From there it can be displayed
to an interactive window or saved to disk and be used like any other
image file.  However, there is currently no good way to ``reopen'' a
Matplotlib \texttt{Figure} or export it to another plotting library,
such as \texttt{bokeh}, \texttt{d3} or \texttt{QtCharts}.
Further, due to the way Matplotlib internals are implemented, it
is difficult to take advantage GPUs to accelerate drawing.

To address these problems we will investigate adding two additional
export paths.  One at a high-level, suitable for a Matplotlib-specific
file format and interoperability with other high-level plotting
libraries, and one at a low level scene-graph level, suitable to pass
to a GPU.


\subsection{Coordination with downstream projects}

This means there will
always be a need for domain specific visualization libraries.

Much of the domain-specific specialization is carried in the semantics
of the structured data and the specific visualization needs of the
domain and the most common visualizations in a domain need to be one or
two simple lines of code for the end-practitioners with the ``obvious''
customization options need to be surfaced.  Our goal is to make these
libraries as thin and easy to write as possible.


To this end we will identify and engage with downstream libraries in
the life sciences that are currently using Matplotlib for their
visualization to identify their pain points and ensure that we are
actually solving their problems.  In particular we plan to engage with
\texttt{scikit-learn}\footnote{Also applying for Essential Open Source
Software for Science}, \texttt{CellProfiler}\footnote{Currently funded
by CZI\label{f:czi}}, \texttt{scanpy}\footref{f:czi},
\texttt{starfish}\footref{f:czi}, \texttt{nipy}, and
\texttt{scikit-image}\footref{f:czi}


\section{Expected outcomes, success evaluation and metrics}
\subsection{Issue and PR curation}

Quantitatively evaluating maintenance work can be tricky, some Issues
or PRs can take minutes to review where as others can take days to
months of effort, however we believe that there is value at looking at
the net number of new Issues and Pull requests.  We will reduce this
number, ideally making it negative.  NumPy has had success in
reversing the ever increasing trends in the number of Issues / Pull
Requests with paid
developers \footnote{https://github.com/seberg/numpy\_talk\_plots/blob/master/plots\_used\_in\_talk/issues\_prs\_delta.pdf}.


We will evaluate and label every open Issue and Pull Request
determining: assigning an action, a priority, and an estimated
difficulty.  Once that is done, we will aim to have all new Issues and
Pull Requests labeled with in 7 days of being opened.


\subsection{Road-map and Architecture}

We will write a white paper and road map documenting the proposed
design, critical use-cases, and requirements for the data model and
API overhaul.  We will and develop prototypes of the end-to-end use
targeting one or more of the life-science libraries discussed above.



\section{Work Plan}

The funds will be paid to:


\begin{itemize}

\item Fund Thomas Caswell's position for 6 months.  Caswell is
  currently the lead developer of Matplotlib and an Associate
  Computational Scientist at Brookhaven National Laboratory.  His
  long-term experience, API design expertise, and project leadership
  are critical to the success of the work in this proposal.  He will work
  on all aspects of the proposal.
\item Fund Hannah Aizenman's position for 12 months.  Aizenman has
  been a core-contributor Matplotlib for three years and has
  previously contributed support for string-categorical values.  She
  is a PhD candidate in computer science studying visualization at The
  City College of New York.  Her work on the architecture of
  Matplotlib will be the basis of her PhD thesis.  Aizenman will take
  the lead on the data model design and new-contributor on-boarding.
\item Fund 12 months of a yet-to-be identified software engineer to
  support all aspects of the proposal but focusing on maintenance,
  prototyping, and engaging down-stream libraries.
\item Travel to key Scientific and Python conferences (such as SciPy
  or PyCon) and for in-person meetings if required.
\end{itemize}

We want to use this dedicated effort to leverage and empower the
Matplotlib developer community.  In terms of direct work on the code
base an equal amount of time will be spent mentoring and reviewing
code from community members rather than directly implementing features
or fixing bugs.  All of the design work will be done in public with
input from the community.

Part of this work is to develop the project road-map.


\section{Existing Support}

Thomas Caswell has 4hrs/wk from Brookhaven National Lab to work on
Matplotlib.


\clearpage
\bibliographystyle{alpha} % or named ?
\bibliography{biblo}

\end{document}
