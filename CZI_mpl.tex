\documentclass[11pt]{article}  % min is 11pt

\usepackage[margin=1in]{geometry}  % min is .5 in
\usepackage[shortlabels]{enumitem}
\usepackage{graphicx}
\usepackage[hidelinks]{hyperref}
\usepackage{caption}




\begin{document}
\title{​​Matplotlib - Foundation of Visualization in Python}
\author{}
\maketitle


The purpose of this proposal is to enable Matplotlib to continue
serving as the primary plotting library of the scientific Python
ecosystem.  We will respond to the backlog of issues and suggestions,
and plan Matplotlib's evolution to meet the challenges of the coming
decade.

By most measures Matplotlib is a wildly successful project.  The
source has been actively developed and maintained by a vibrant
community over the last 16 years.  Conservatively we have 1M monthly
users across the wide spectrum of fields, including bio-medical
imaging, microscopy, and genomics
\cite{Carpenter2006,Wolf2018,10.7717/peerj.453}
\cite{Segata2011,10.1371/journal.pgen.1000695,HASHIMSHONY2012666,
  10.1093/bioinformatics/bts480,Carlile2014,Laganowsky2014,Jiangaac9462,
  10.3389/fninf.2014.00014}.
Our user base continues to grow as Python is adopted by more
scientists in the life sciences.  However, given the scale and scope
of the project, we are at the limit of what can be maintained and
developed with primarily volunteer effort. This proposal identifies
three key areas where we need sustained support to ensure the health
of the platform and provide the leadership to meet the visualization
challenges of the next 16 years:

\begin{enumerate}[label=\alph*)]
  \item Maintenance of the library by curating existing and new Issues
    and Pull Requests.
  \item Developing a comprehensive plan to update the core Architecture
    of Matplotlib
  \item Developing the tools, documentation, and community to support
    a rich eco-system of domain specific plotting tools built on top
    of Matplotilb.
\end{enumerate}

We propose to fund Thomas Caswell (PI) for 6 months, Hannah Aizenman
for 12 months, and a third yet-identified software engineer to work on
these tasks.  Caswell is the lead developer of Matplotlib and an
Associate Computational Scientist at Brookhaven National Laboratory.
Aizenman is a core-contributor to Matplotlib who has previously worked
on datatypes and outreach, and a PhD student in computer science
studying visualization at The City College of New York.


\section{Goals}
\subsection{Curating Pull Requests and Issues}

The rate at which new Issues and Pull Requests come into the project
is out-pacing the rate at which volunteer effort can review them.
Over the past few years we have continued to merge PRs and close
issues about about [X per month], however new issues and PRs are opened
at a rate of [Y per month] which has produced a significant backlog.
This has resulted in useful contributions languishing in un-merged PRs
and bugs that break downstream packages going un-addressed.  This has
a deleterious effect on our community, discouraging new contributors,
and making it challenging to develop new contributors to the level of
familiarity with the code that they are able to review PRs. This leads
to an obvious negative feedback loop, where the community does not
grow in a healthy and sustainable manner.

The developer time asked for here will partly go towards active and
sustained curation of issues and pull requests.  By having a small
team working at least half time on Matplotlib we can ensure that :

\begin{itemize}
\item old issues are dealt with and new issues receive quick engagement
\item new contributors are effectively on-boarded to the project team
\item backward compatibility is maintained as much as possible and
  that when we do break backward compatibility it is intentional and
  well documented
\item group decisions as to whether proposed enhancements and features
  should be implemented will be better managed
\end{itemize}

None of this is to demote the importance of the volunteer
contributors, but it will lead to a better co-ordination and nurturing
of their efforts, with the goal of growing a diverse community of
expert contributors.

\subsection{Roadmap and Architecture Design}

The current architecture of Matplotlib was first designed 15 years ago
\cite{Hunter:2007}.  That it is still in use and alive is a testament
to its initial design, however it does not reflect all of the
developments in thinking around data structures, software design, and
visualization of the past decade.  Matplotlib does not natively know
how to exploit structured data (e.g. \texttt{pandas} or
\texttt{xarray}) objects or to seamlessly handle streaming data.

\subsubsection{Data Model}

Structured data combines one or more data sets into an object that
also contains dimensions and co-ordinates of the data (i.e. think of a
map of global surface temperatures containing $T$ , $latitude$ and
$longitude$ being packed together in an object along with all the
relevant meta data).  Currently Matplotlib asks the users or
downstream libraries to split this information into its components and
pass them individually to plotting methods, losing the structure of
the data.  User data is

Similarly, Matplotlib has support for handling data with associated
units (i.e. degrees Celsius), but this has been a difficult due to
given the many places where data may be stored, and hence will need to
handle units, throughout the package.

Similarly, the way user data is
stored internally makes it challenging to make interactive, streaming,
and animated visualizations.

The user must know the details about how
to update each of the individual elements.c

\subsubsection{Interactivity and Streaming}
\subsubsection{Homogenization of the API}

\subsection{Coordination with downstream projects}

\section{Work Plan}
The funds will be paid to:

\begin{itemize}

\item Fund Thomas Caswell's position for 6 months.  Caswell is
  currently the lead developer of Matplotilb.  His long-term
  experience, API design expertise, and project leadership are critical
  to the success of the work in this proposal.
\item Fund Hannah Aizenman's position for 12 months.  Aizenman has
  been a core-contributor Matplotlib for three years and has
  previously contributed support for string-categorical values.  Her
  work on the design of Matplotlib will be the basis of her PhD
  thesis.  Aizenmen will take the lead on new-contributor on-boarding.
\item Fund 12 months of a yet-to-be identified software engineer to
  support all aspects of the proposal.  This may also cover travel to
  NYC for meetings if the third developer is remote.
\item Travel to key Scientific and Python conferences (such as SciPy
  or PyCon)
\end{itemize}

We want to use this dedicated effort to leverage and empower the
Matplotlib developer community.  In terms of direct work on the code
base an equal amount of time will be spent mentoring and reviewing
code from community members rather than directly implementing features
or fixing bugs.  All of the design work will be done in public with
input from the community.

Part of this work is to develop the project roadmap.


\section{Existing Support}

Thomas Caswell has 4hrs/wk from Brookhaven National Lab to work on Matplotlib.


\clearpage
\bibliographystyle{alpha} % or named ?
\bibliography{biblo}

\end{document}
