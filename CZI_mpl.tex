\documentclass[11pt,letterpaper]{article}  % min is 11pt

\usepackage[sort&compress,numbers]{natbib}
\usepackage[margin=1in,letterpaper]{geometry}  % min is .5 in
\usepackage[shortlabels]{enumitem}
\usepackage{graphicx}
\usepackage{caption}
\usepackage{footmisc}
\usepackage[hidelinks]{hyperref}

\linespread{1.5}

\begin{document}
\title{Matplotlib - Foundation of Visualization in Python}
\author{}
\maketitle

\section{Goals}

Matplotlib\footnote{\url{http://matplotlib.org}} is the fundamental
data visualization library for the scientific Python ecosystem, used
by over a million\footnote{Estimated from \texttt{pypi} download
numbers, \texttt{conda} download numbers, and the number of unique
monthly visitors to the documentation website} users in conjunction
with other foundational tools like Numpy and
SciPy\footnote{\url{https://www.scipy.org/about.html}}.  Matplotlib is
used across a wide spectrum of fields, including bio-medical imaging,
microscopy, and genomics
\cite{Carpenter2006,Wolf2018,10.7717/peerj.453,
  Segata2011,10.1371/journal.pgen.1000695,HASHIMSHONY2012666,
  10.1093/bioinformatics/bts480,Carlile2014,Laganowsky2014,Jiangaac9462,
  10.3389/fninf.2014.00014}, and we expect this user base to continue
to grow as Python is further adopted in the life sciences.

Matplotlib has been actively developed and maintained by a vibrant,
primarily volunteer, community over the last 16 years.  However, given
the scale, scope, and importance of the project, we are at the limit
of what we can sustainable maintain and develop with primarily
volunteer effort.

This proposal asks for 2.4 person years of developer effort for
Matplotlib, laying the groundwork for the next 16 years by supporting:

\begin{enumerate}[label=\alph*),noitemsep]
  \item Maintenance of the library, including curating new and
    existing Issues and Pull Requests (PRs).
  \item Developing a comprehensive plan to evolve the core architecture
    of Matplotlib.
  \item Developing the tools, documentation, and community to foster a
    rich eco-system of domain-specific plotting tools built on
    Matplotlib.
\end{enumerate}


% The goal is the maintenance; curating is the method of achieving it.
\subsection{Maintenance and Growth of Library and Community}

% JMK Try to start paragraphs with what you will do, and then defend
% why it is necessary
% JMK I'm pretty skeptical that CZI will find closing issues sexy
% enough.  Thats why I was suggesting curation and community-building
% as a way to express the same thing.

Matplotlib is a community-driven project, but we have grown to the
point where we need supported developers with the time to organize,
plan, and make timely decisions.  Currently, new Issues and PRs are
being submitted faster than our volunteers can review them.  Over the
past few years we have merged PRs and closed Issues at about [X per
  month], but about [Y] new issues and PRs are opened monthly;
currently we have about 1200 open issues and 300 open PRs.  Among the
former there may be critical bug reports or insightful feature
requests and among the latter are useful contributions or bug fixes
that would improve Matplotlib for direct users and downstream
packages.  The backlog is discouraging for new or occasional
contributors, and distracting for core developers.

To maintain Matplotlib's health we need to:

\begin{itemize}[noitemsep]
\item Curate the current backlog of Issues and PRs
  in terms of topic, difficulty, and urgency.
\item Label and review newly opened Issues and PRs
  promptly.
\item Fix critical bugs and regressions promptly.
\item On-board new contributors to the project team; this is critical to
  sustaining and diversifying our developer community.
  % AL had concerns about this being an agressive growth target, I think
  % of this more in terms of making sure we are bring people on to offset
  % the people eventually rolling off (for what ever reason)
  % JMK: We hav a backlog of PRs largely because we don't have enough
  % people to review them. I think a larger community is key.
\item Maintain backward compatibility.  If we do break API, ensure it
  is intentional and well documented.
\item Manage discussions about proposed enhancements, features, and
  breaking API changes.
\end{itemize}

The requested support for developers is intended to complement and
facilitate, not replace, the crucial work by volunteers.  We aim to
better co-ordinate and nurture their
efforts, with the goal of growing and sustaining a diverse community
of expert contributors, both volunteer and paid.  We view
this proposal as the start of a new phase for Matplotlib, in which we
will need to find support from a variety of sources in the coming years.
% Realistically, don't we need *some* funding for paid work to
% continue after this?  I don't think this is a one-shot cure-all.
% Agree, this is at best a jump start / down payment on future funding

% Cite Ipython, Jupyter, Numpy as our models or inspiration in this?


\subsection{Road-map and Architecture}

Matplotlib needs sustained attention to update the library's
architecture for the next decade.  The current
architecture\footnote{\url{https://www.aosabook.org/en/matplotlib.html}}
of Matplotlib was developed 15 years ago \cite{Hunter:2007}.  That it
is still in use is a testament to its initial design; but that design
does not reflect recent developments in data structures, software
design, and visualization.  Matplotlib does not natively know how to
exploit structured (e.g. \texttt{pandas} or \texttt{xarray}) and some
of the design choices about data structures and name space
organization are becoming constraints as Matplotlib is adopted by more
domains.

%% We will develop
%% plans for refactoring the core library to facilitate its extension and
%% to smooth interoperability among the core library and the various
%% domain-specific plotting tools.
%% We will investigate how to update Matplotlib's
%% internal data model to use the information embedded in structured
%% data and allow fluid updating.

\subsubsection{Homogenization of the Application Programming Interface (API)}
\label{sec:api_hom}
% should we merge this into the next section?
The library has grown organically over time through the contributions
of many people (approx. 900 individuals), and the code has accumulated
many small inconsistencies in the API.  Similar methods have different
argument order, e.g., \texttt{ax.text(x, y, s)} vs
\texttt{ax.annotation(s, (x, y))}, and some keyword arguments can be
singular or plural, e.g., \texttt{color} vs \texttt{colors}.  These
subtle issues add friction for users, but are hard to fix without
breaking existing code somewhere in our large user base.  Our goal is
to minimize breakage, but still unify the API.  Taking into account
\textbf{all} of the APIs, we will carefully consider which to leave as
they are, which to deprecate, and which to replace.

\subsubsection{API generalization}
\label{sec:api_gen}
Currently Matplotlib has two main user-facing APIs: the
\texttt{pyplot} API and the \texttt{Object Oriented} (\texttt{OO})
API.  The \texttt{pyplot} API closely follows MATLAB and is built
around the concept of a global ``current Axes''.
While
\texttt{pyplot} is convenient for interactive usage, the global
state frequently produces surprising and undesired coupling when used
in libraries.
% Something is missing in the line above, but I don't know what it is...
On the other hand the \texttt{OO} interface is more flexible and
explicit but more verbose.
The main name space for plotting methods in the \texttt{OO} API is
methods on the \texttt{Axes} class, which leads to three issues with
the API.
First, third party domain-specific packages can never feel ``First
Class'', as their plotting functions will not be implemented as
\texttt{Axes} methods like the ``built in'' ones.
Second, some visualizations require putting \texttt{Artist}s on
multiple \texttt{Axes}; this doesn't fit the model in which Artists
are made via \texttt{Axes} methods.
Lastly, there are over 250 methods on the \texttt{Axes}---all of the
plotting methods and some additional \texttt{Axes} specific
methods---which makes it extremely hard to use tab-completion to
search for a method.

To address these issues we will move to a primary API in which
top-level functions take in data, style, and \texttt{Axes}(s) and
return \texttt{Artist}s.  These functions will use consistent naming
and call conventions, as discussed in \ref{sec:api_hom} and return the
\texttt{Artist}s discussed in section \ref{sec:artists}.  From the
large pool of plotting functions, augmented by downstream libraries,
we can curate domain-specific name spaces to enable users to discover
the functionality they need.

This large-scale refactoring requires dedicated developer
time to carefully consider the consequences and trade offs.

% This is a huge change in the API that will require time to carefully
% consider the consequences.
% --
% Does this mean it the consequences might be so bad that it won't be
% pursued?  My point is that the sentence is true--but its role in the
% proposal is unclear to me.  I'm not sure it is what you really want
% to say here.



\subsubsection{Rich, Semantic \texttt{Artist}s}
\label{sec:artists}
\texttt{Artist}s are the ``middle layer'' of Matplotlib that encode
user-intent, style, and data.
To update the style or data users interact with the \texttt{Artist}
objects, and \texttt{Artist}s know how to
translate that intent on to the rendered output.

We need to embrace composite artists
to provide a smarter, more uniform \texttt{Artist} API.
Currently, Matplotlib has a mix of ``primitive'' \texttt{Artist}s
(e.g. lines, images, and text) and ``composite'' artists (e.g.,
the whole \texttt{Figure}).
Presently the mapping between the user API and the \texttt{Artist}s can be
one-to-one, but often one user call will generate many decoupled
primitive \texttt{Artist}s.
For example, \texttt{hist} displays a histogram,
but returns a list of independent \texttt{Rectangle} \texttt{Artist}s
(one per bin).
If the user wants to update the data or the style,
they must adjust each \texttt{Rectangle} independently.
Instead, the
functions discussed in \ref{sec:api_gen} should return objects with
a uniform interface for updating their data and style, making
interactive visualizations easier for all users to develop.




\subsubsection{Data Model}
\label{sec:dm}

Matplotlib needs a consistent way for \texttt{Artist}s to access their
underlying data.
Further, each plotting method and \texttt{Artist} handles sanitizing
and storing data independently.
Hence some common functionality, such as handling data with attached
units (e.g., degrees Celsius, dates), is scattered throughout the code
base.
These lead to inconsistencies across the library and makes it
difficult to write code that updates the data or style for interactive
exploration and animation.


``Structured data'' combines multiple pieces of, possibly
heterogeneous, data along with labels and metadata into single data
structure, is revolutionizing science, but is not natively supported by
Matplotlib.
Instead, extracting
data elements and reassembling them as arguments for Matplotlib plotting
functions must be done at the level of user code.



We will develop a data access API such that each \texttt{Artist} will
have a \texttt{Data} object responsible for actually holding the data.
This model will support the base Matplotlib library and, more
importantly, will be the technical underpinning to handle, exploit, and
update structured data in a coherent fashion.
By removing the direct data storage from
the \texttt{Artist}s and defining an API for data sources we will enable:
\begin{itemize}[noitemsep]
  \item native consumption of structured data;
  \item smart down-sampling of plotted data based on view limits;
  \item seamless updating of the underlying data, either
    streaming or interactively;
  %% this one maybe too much, cool idea, but not really mentioned anywhere else
  \item use of alternative data sources such as database queries or analytic functions
\end{itemize}
This will decouple the development of the data sources from the
\texttt{Artist}s.  Downstream libraries will be able to provide
sophisticated data sources to the \texttt{Artists} in the core library
or sophisticated that \texttt{Artist}s that use the data sources from
core.


% \subsubsection{Additional Export Methods}
% % maybe nuke this section?
% Matplotlib \texttt{Figure}s can render to either raster or vector
% file formats, displaying the result in an interactive window and/or
% saving it to disk.  There is
% currently no good way to save and reopen a Matplotlib \texttt{Figure} or
% export it to another plotting library, such as \texttt{bokeh},
% \texttt{d3} or \texttt{QtCharts}.  In addition, due to the way Matplotlib
% internals are implemented, it is difficult to take advantage of GPUs to
% accelerate drawing.
%
% To address these problems we will investigate adding two additional
% export paths.  One, at a high-level, will be suitable for subsequent re-input
% to Matplotlib
% and for interoperability with other high-level plotting
% libraries. The second, at a low scene-graph level, will be suitable for passing
% to a GPU.
%
%

\subsection{Coordination with downstream projects}


The most common visualizations in a domain need to be fluid
for the end-practitioners, with the ``obvious''
customization options exposed.
Much of the domain-specific
specialization carried in the semantics of the structured data and the
specific visualization needs of the domain.  These vary widely between domains;
no library can meet all of the needs simultaneous so there will always
be a need for domain-specific visualization libraries.
Our goal is to make these libraries as easy to write as possible.
% That last sentence still needs work.
We will develop a set of guidelines to for the APIs so that downstream
libraries ``feel'' similar and inter-operate well as well as tools and
templates to make the process of developing a new library painless.

We will engage with libraries in the life sciences that are currently
using Matplotlib for their visualization to ensure that we are
addressing their problems and to collaborate on prototyping a
end-to-end solution.
In particular we plan to engage with
\texttt{scikit-learn}\footnote{Also applying for Essential Open Source
Software for Science}, \texttt{CellProfiler}\footnote{Currently funded
by CZI\label{f:czi}}, \texttt{scanpy}\footref{f:czi},
\texttt{starfish}\footref{f:czi}, \texttt{nipy}, and
\texttt{scikit-image}\footref{f:czi}




\section{Expected outcomes, success evaluation and metrics}
\subsection{Issue and PR curation}

Quantitatively evaluating maintenance work can be tricky---some Issues
or PRs can take minutes to review while  thers can take days to
months of effort---but we believe that there is value at looking at
the number of open Issues and PRs.  We will reduce this
number, closing Issues and PRs faster than they are
opened until a reasonable equilibrium is reached.
With the introduction of paid developers,
NumPy has had success in reversing the ever increasing trends
in the number of Issues and Pull Requests\footnote{https://github.com/seberg/numpy\_talk\_plots/blob/master/plots\_used\_in\_talk/issues\_prs\_delta.pdf}.


We will evaluate and label every open Issue and Pull Request to
assign an action, a priority, and an estimated
difficulty.  Once the backlog is handled, we will aim to have all new Issues and
Pull Requests labeled within 7 days of being opened.


\subsection{Road-map and Architecture}

We will write a white paper with a road map documenting the proposed
design, critical use-cases, and requirements for the data model and
API overhaul.

To validate the design, we will develop end-to-end prototypes
targeting one or more of the life-science libraries discussed above
to validate the design.

% JMK Does this section need more?
% EF Would the prototyping just show what use of the API would look like?
%    How much of the actual mpl implementation would be expected?
%    What is actually feasible in a year?
%    It is OK for the proposal to be ambitious, but it should also
%    show some recognition of reality.

\section{Work Plan}

We request funding for the following:


\begin{itemize}[noitemsep]

\item Fund Thomas Caswell's position at 40\%.  Caswell is
  currently the Lead Developer of Matplotlib and an Associate
  Computational Scientist at Brookhaven National Laboratory.  His
  long-term experience, API design expertise, and project leadership
  are critical to the success of the work in this proposal.  He will work
  on all aspects of the proposal.
\item Hannah Aizenman's position for 12 months.  Aizenman has
  been a core-contributor Matplotlib for three years and is the author
  of a major recent feature set (support for string-categorical values).  She
  is a PhD candidate in computer science studying visualization at The
  City College of New York.  Her work on the architecture of
  Matplotlib will be the basis of her PhD thesis.  Aizenman will take
  the lead on the data model design and new-contributor on-boarding.
\item 12 months of a yet-to-be identified software engineer to
  support all aspects of the proposal but focusing on maintenance,
  prototyping, and engaging down-stream libraries.
\item Travel to key Scientific and Python conferences (such as SciPy
  or PyCon) and for in-person meetings if required.
\end{itemize}

We want to use this dedicated effort to leverage and empower the
Matplotlib developer community.  In terms of direct work on the code
base, equal amounts of time will be spent mentoring and reviewing
code from community members, and directly implementing features
or fixing bugs.  All of the design work will be done in public with
input from the community.

Part of this work is to develop the project Road Map.



\section{Existing Support}

Thomas Caswell has 4hrs/wk from Brookhaven National Lab to work on
Matplotlib.


\clearpage
\bibliographystyle{ieeetr} % or named ?
\bibliography{biblo}

\end{document}
