\documentclass[11pt]{article}  % min is 11pt

\usepackage[margin=1in]{geometry}  % min is .5 in
\usepackage[shortlabels]{enumitem}
\usepackage{graphicx}
\usepackage{caption}
\usepackage{footmisc}
\usepackage[hidelinks]{hyperref}



\begin{document}
\title{Matplotlib - Foundation of Visualization in Python}
\author{}
\maketitle

\section{Goals}

The purpose of this proposal is to plan Matplotlib's evolution to meet
the visualization challenges of the coming decade and reduce backlog
of currently open Issues and Pull Requests.

By most measures Matplotlib is a wildly successful project; the source
has been actively developed and maintained by a vibrant primarily
volunteer community over the last 16 years and, conservatively, has
over a million users across the wide spectrum of fields, including
bio-medical imaging, microscopy, and genomics
\cite{Carpenter2006,Wolf2018,10.7717/peerj.453}
\cite{Segata2011,10.1371/journal.pgen.1000695,HASHIMSHONY2012666,
  10.1093/bioinformatics/bts480,Carlile2014,Laganowsky2014,Jiangaac9462,
  10.3389/fninf.2014.00014}.  We expect our user base to continue to
grow as Python is adopted by more scientists in the life sciences.
However, given the scale, scope, and importance of the project, we are
at the limit of what can be developed and maintained with primarily
volunteer effort.  This proposal identifies three key areas where we
need sustained support to ensure the health of the platform and
provide the leadership to meet the visualization challenges of the
next 16 years:

\begin{enumerate}[label=\alph*)]
  \item Maintenance of the library by curating existing and new Issues
    and Pull Requests.
  \item Developing a comprehensive plan to evolve the core architecture
    of Matplotlib.
  \item Developing the tools, documentation, and community to foster a
    rich eco-system of domain-specific plotting tools built on
    Matplotlib.
\end{enumerate}


% This matches the scikit-learn structure, but to my eye it doesn't
% belong here at all.  This is not a goal, but we are in the goals
% section.  This belongs in the work plan section.
We propose to fund Thomas Caswell (PI) for 6 months, Hannah Aizenman
for 12 months, and a yet-unidentified software engineer for 12 months to
work on these tasks.  Caswell is the lead developer of Matplotlib and
an Associate Computational Scientist at Brookhaven National
Laboratory.  Aizenman is a core contributor to Matplotlib, who has
previously worked on datatypes and outreach; she is a PhD student in
computer science studying visualization at The City College of New
York.


% The goal is the maintenance; curating is the method of achieving it.
\subsection{Maintenance of the library}

New Issues and Pull Requests are being submitted faster than our
volunteers can review them.
Over the past few years we have merged PRs and closed
issues at about about [X per month], but about [Y] new issues and PRs
are opened monthly; currently we have about 1200 open issues and
300 open PRs.  Among the latter are good contributions and bug fixes
that, possibly with additional attention and polish, could improve
matplotlib for direct users and downstream packages.  The backlog is
discouraging for new or occasional contributors, and distracting for
core developers.

To maintain matplotlib's health we need to do the following:

\begin{itemize}
\item Sort the current backlog of Issues and Pull Requests
  in terms of urgency and difficultly.
\item Ensure that newly opened Issues are sorted and Pull Requests
  are reviewed in a timely manner.
\item On-board new contributors to the project team.  This is
  critical to sustaining and diversifying our developer community.
\item Maintain backward compatibility as much as practical.  If we
  do break API, we must ensure it is intentional and well documented.
\item Manage group decisions about proposed enhancements, features, and
  breaking API changes.
\end{itemize}

None of this is to demote the importance of the volunteer
contributors, but instead to better co-ordinate and nurture their
efforts, with the goal of growing and sustaining a diverse community
of expert contributors.
% Realistically, don't we need *some* funding for paid work to
% continue after this?  I don't think this is a one-shot cure-all.

\subsection{Road-map and Architecture Design}

The current
architecture\footnote{https://www.aosabook.org/en/matplotlib.html} of
Matplotlib was developed 15 years ago \cite{Hunter:2007}.  That
it is still in use is a testament to its initial design; but that
design does not reflect more recent developments in data
structures, software design, and visualization.
Matplotlib does not natively know how to exploit structured data
(e.g. \texttt{pandas} or \texttt{xarray}) objects.  We will investigate
how to update Matplotlib's internal data model to use the
information embedded in structured data.  While there are downstream
domain-specific libraries that are built on top of Matplotlib,
interoperability is problematic.  We will develop plans for
refactoring the core library to facilitate its
extension and to smooth interoperability among the core
library and the various domain-specific plotting tools.

We will develop clear requirement documents, document critical use
cases, implement prototypes, and engage with downstream users to
ensure that both the data model and API overhauls enable domain-specific
plotting tools to be easily built on top of Matplotlib.


\subsubsection{Data Model}

``Structured data'' combines one or more arrays into an object that
also contains labels, dimensions, and co-ordinates of the data (e.g., a
map of global surface temperatures containing \texttt{T} ,
\texttt{latitude} and \texttt{longitude} being packed together in an
object along with all the relevant metadata).  In addition to the
data itself, these objects carry information about the
relationship, implicitly or explicitly, between the components.
Currently Matplotlib asks the users or downstream libraries to split
this information into its components and pass them individually to
plotting methods, losing the structure and relationships.
Internally, each plotting method (and \texttt{Artist}s, the objects
used to represent the plot elements) each handle sanitizing and storing user
input independently.  This means that some common functionality, such
as handling data with attached units (e.g., degrees Celsius), is scattered
throughout the
code base.  This leads to inconsistencies in behavior across the
library and makes it difficult to write code that updates the user
data, as is required for interactive exploration, streaming, and animation
use cases.


Here we propose to put significant effort into re-designing the
internal data model for Matplotlib, with a prototype implemention. The
goal is to have a simple model appropriate for the base Matplotlib
library and, more importantly, to have the technical underpinning in
place that will allow domain-specific downstream libraries to
handle and update structured data in a coherent fashion.  By
removing the direct data storage from the \texttt{Artist}s and
defining an API for data objects we will make it easy to:
\begin{itemize}
  \item implement smart down sampling of plotted data based on view
    limits;
  \item seamlessly update the underlying data, either
    streaming or interactively;
  \item and back the plot with non-numpy arrays, including queries to a
    database.
\end{itemize}
By defining a clear API to access data we will be able to decouple
the development of the data access from the \texttt{Artist}s.  This
will enable implementations of the data layer that have exotic
dependencies to be used with \texttt{Artist}s from the core library, and
third-party \texttt{Artist}s to rely on the data layer from the core
library.

\subsubsection{Homogenization of the API}

The library has grown organically over time through the contributions
of many people (approximately 900 individuals) and the code has
accumulated many small inconsistencies in the API.  Similar methods
have different argument order, e.g., \texttt{ax.text(x, y, s)} vs
\texttt{ax.annotation(s, (x, y))}, and some keyword arguments can be
singular or plural, e.g., \texttt{color} vs \texttt{colors}.  These
subtle issues add friction for users, but are hard to fix without
breaking existing code somewhere in our large user base.
Our goal is to minimize breakage.  Taking into account
\textbf{all} of the APIs, we will have to carefully consider
which to leave as they are, which to deprecate, and which to
replace.


\subsubsection{Rich Composite \texttt{Artist}s}

\texttt{Artist}s are the ``middle layer'' of Matplotlib that we use to
encode user-intent and data which can ``draw'' themselves when
displaying a figure to the screen or saving to disk.  These are the
objects that users interact with to update the plotted data or style.
Currently, Matplotlib has a mix of ``primitive'' \texttt{Artist}s (for
things like lines, images, and patches) and ``composite'' artists for
things like the whole \texttt{Figure}.  In some cases the mapping
between the user API and the resulting \texttt{Artist}s is one-to-one,
but in other cases one user call may generate many de-coupled
\texttt{Artist}s, for example, \texttt{errorbar} creates up to 5
independent \texttt{Artist}s.  If the user wants to update the plot
they must touch all of the objects.  Instead, we should be providing
the user with a single object on which they can interact with to elide
the underlying details of how the \texttt{Artist} is rendered.  This
will greatly simplify user code for interactive figures, streaming
data, and animations.


\subsubsection{API generalization}

Currently Matplotlib has two main user-facing APIs: the
\texttt{pyplot} API and the \texttt{Object Oriented} (\texttt{OO})
API.  The \texttt{pyplot} API closely follows MATLAB and, while
convenient for quick interactive usage, it is built around the concept
of a global ``current Figure'' and ``current Axes''.  This is
problematic if used in down stream libraries by leading to surprising
and frequently undesired coupling between different parts of the code.
On the other hand the \texttt{OO} interface is more explicit and
flexible but marginally more verbose.  However, the main name space
for plotting methods is on the \texttt{Axes} object which leads to
three issues with the API.  First because the plotting methods
provided by Matplotlib are methods on one of our classes third party
domain-specific packages can never feel ``First Class'' as they will
not be implemented as \texttt{Axes} methods.  Second, there are some
plots that should be easy that require putting \texttt{Artist}s on
multiple \texttt{Axes} which is not something that can be naturally
expressed as an \texttt{Axes} method.  Lastly, because all of the
plotting methods (along with some additional \texttt{Axes} specific
methods) are in a single name space, there are over 250 methods on the
\texttt{Axes} class which makes it extremely hard to discover if the
method you need exists.

To address both of these issues we will move the main plotting name
space from the \texttt{Axes} method to top-level functions that take in
data, style, and \texttt{Axes}s.  This will allow ``first-party'' and
``third-party'' plotting tools to be on a level playing field and
makes it possible to cleanly write plotting tools that produce
multiple \texttt{Artist}s across multiple \texttt{Axes}s.  These
functions will return the rich composites discussed in the previous
section.  Having all of the plotting functions, both first- and
third-party, as top level functions allows for curated domain-specific
name spaces.  This will greatly aid in the discoverabilty of relevant
plotting functions by users.



\subsubsection{Additional Export Methods}

The primary way to export a Matplotlib figure is to save it in either
as raster or vector file format.  From there it can then be opened in
other applications, emended in papers or websites, or displayed to an
interactive window.  However, there is currently no way good way to
save and reopen a Matplotlib \texttt{Figure} or export it to another
plotting library, such as \texttt{bokeh}, \texttt{d3} or
\texttt{QtCharts}.  Additionally, due to the way Matplotlib internals
are implemented, it is difficult to take advantage GPUs to accelerate
drawing.

To address these problems we will investigate adding two additional
export paths.  One at a high-level, suitable for a Matplotlib-specific
file format and interoperability with other high-level plotting
libraries, and one at a low level scene-graph level, suitable to pass
to a GPU.


\subsection{Coordination with downstream projects}

Much of the domain-specific specialization is carried in the semantics
of the structured data and the specific visualization needs of the
domain.  The most common visualizations in a domain need to be one or
two simple lines of code for the end-practitioners and the ``obvious''
customization options need to be surfaced.  This means there will
always be a need for domain specific visualization libraries.  All of
the refactoring describe above is motivated to enable these libraries
to be easily written and extended.  To this end we will engage with
current down stream projects to make sure that our plans actually to
make things easier for them and focus our prototyping efforts on their
use cases.  In particular we plan to engage with
\texttt{scikit-learn}\footnote{Also applying for Essential Open Source
Software for Science}, \texttt{CellProfiler}\footnote{Currently funded
by CZI\label{f:czi}}, \texttt{scanpy}\footref{f:czi},
\texttt{starfish}\footref{f:czi}, \texttt{nipy}, and
\texttt{scikit-image}\footref{f:czi}


\section{Expected outcomes, success evaluation and metrics}
\subsection{Issue and PR curation}

Quantitatively evaluating maintenance work can be tricky, some Issues
or PRs can take minutes to review where as others can take days to
months of effort, however we believe that there is value at looking at
the net number of new Issues and Pull requests.  We will reduce this
number, ideally making it negative.  NumPy has had success in
reversing the ever increasing trends in the number of Issues / Pull
Requests via paid
developers \footnote{https://github.com/seberg/numpy\_talk\_plots/blob/master/plots\_used\_in\_talk/issues\_prs\_delta.pdf}.


We will also evaluate and label every open Issue and Pull Request
determining: assigning an action, a priority, and an estimated
difficulty.  Once that is done, we will aim to have all new Issues and
Pull Requests labeled with in 7 days of being opened.


\subsection{Architecture and down stream engagement}

For the Architecture work we plan to produce a white paper documenting
the proposed design, document the use-cases that need to be supported,
and develop prototypes of the end-to-end use targeting one or more of the
life-science domains.


\section{Work Plan}

The funds will be paid to:

\begin{itemize}

\item Fund Thomas Caswell's position for 6 months.  Caswell is
  currently the lead developer of Matplotlib.  His long-term
  experience, API design expertise, and project leadership are critical
  to the success of the work in this proposal.
\item Fund Hannah Aizenman's position for 12 months.  Aizenman has
  been a core-contributor Matplotlib for three years and has
  previously contributed support for string-categorical values.  Her
  work on the architecture of Matplotlib will be the basis of her PhD
  thesis.  Aizenman will take the lead on the data model design and
  new-contributor on-boarding.
\item Fund 12 months of a yet-to-be identified software engineer to
  support all aspects of the proposal.
\item Travel to key Scientific and Python conferences (such as SciPy
  or PyCon) and for in-person meetings if required.
\end{itemize}

We want to use this dedicated effort to leverage and empower the
Matplotlib developer community.  In terms of direct work on the code
base an equal amount of time will be spent mentoring and reviewing
code from community members rather than directly implementing features
or fixing bugs.  All of the design work will be done in public with
input from the community.

Part of this work is to develop the project road-map.


\section{Existing Support}

Thomas Caswell has 4hrs/wk from Brookhaven National Lab to work on Matplotlib.


\clearpage
\bibliographystyle{alpha} % or named ?
\bibliography{biblo}

\end{document}
